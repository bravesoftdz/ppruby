\documentclass[a4paper,10pt,DIVcalc,final]{scrartcl}

\usepackage[utf8]{inputenc}
\DeclareUnicodeCharacter{00A0}{~}
\usepackage[russian]{babel}

\usepackage[unicode,bookmarksnumbered,pdfview=FitH,pdfstartview=FitH]{hyperref}

\parindent=0pt
\parskip=2pt plus 0.2pt minus 0.1pt
\tolerance=1000

\author{Иван Шихалев}
\title{ppRuby}
\subtitle{использование интерпретатора Ruby в Free Pascal}

\hypersetup{
 pdfkeywords={Ruby, Free Pascal, Lazarus},
 pdfsubject={Программирование},
 pdfauthor={Иван Шихалев},
 pdftitle={ppRuby — использование интерпретатора Ruby в Free Pascal}
}

\begin{document}

\maketitle

\begin{abstract}
\noindent\textsf{\textbf{ppRuby}} — это набор модулей Free Pascal и пакетов 
Lazarus для ис\-поль\-зо\-ва\-ния интерпретатора Ruby в качестве скриптового
движка в pascal-программах.
Проект свободно распространяется под лицензией GNU General Public License и 
может быть найден по адресу \url{https://github.com/shikhalev/ppruby}.
\end{abstract}

\tableofcontents

\section{Обзор}

\section{Использование}

\section{Расширение}

\section{Разработка}

\end{document}